\section{Aristoteles}
Aristoteles beschäftigte sich bereits zu seiner Zeit mit der Natur und Technik. Wenngleich die Technik zu dieser Zeit eine ganz andere Bedeutung als heute hat. Zunächst soll der Begriff Natur, aus seiner Sicht, genauer erläutert werden. Im Buch V der Metapyhsik, auch bezeichnet als \glqq Lexikon der philosphophischen Terminlogie\grqq{}, geht Aristoteles auf die Bedeutung von Natur ein. Als erstgenannte Hauptbedeutung von Natur nennt er
\begin{quote}
	\glqq das Wesen derjenigen Dinge [...], die in sich selbst über das Prinzip der Bewegung verfügen, sofern sie diese Dinge sind, die sie sind\grqq{}
	\cite[1015a14 f.]{MetaphysikV} 
\end{quote}

%Eine weitere Aussage in Metaphysik V lautet:
%\begin{quote}
%	\glqq Prinizip der Bewegung der Dinge, die von Natur aus sind\grqq{}
%	\cite[1015a17 f.]{MetaphysikV} 
%\end{quote}

Wir versuchen die Bedeutung der Zitate in zwei Teilstücke aufzuteilen und zu interpretieren. Erster Teil \glqq das Wesen derjenigen Dinge, die in sich selbst über das Prinzip der Bewegung verfügen\grqq{}, wenn wir aus der heutigen über ein Wesen sprechen, dann ist das soviel wie die grundlegende Eigenart einer Sache. Dazu kommt die Eigenschaft des Wesens über eine Bewegung zu verfügen. Doch was genau beudeutet hier Bewegung? In diesem Fall ist die Bewegung auf Dinge bezogen. Mal angenommen ein Mensch fällt unter die Kategorie Dinge. Dann ist das beim Menschen eigentlich einfach zu sehen, mit dem Heben eines Armes erfolgt eine Bewegung. Vielleicht ist auch mit Bewegung weitaus mehr gemeint. Zum Beispiel die Bewegung der Zellen, die Veränderung der Haut usw.. Doch wenn wir aus heutiger Sicht einen Roboter nehmen. Dann verfügt dieser auch über das Prinzip der Bewegung. Zusätzlich gehört der Roboter zur Kategorie der Dinge. Ist somit auch der Roboter der Natur zuzuordnen? Der zweite Teil des Zitats lautet \glqq sofern sie diese Dinge sind, die sie sind\grqq{}. Steckt hier bereits mehr dahinter als die materielle Bedeutung im Sinne von \glqq das Ding ist ein Roboter\grqq. Kommt hier bereits das Bewusstsein ins Spiel? Sind sich die Dinge bewusst im Sinne von was Sie sind?  

Im ersten Kapitel des zweiten Buches der Physik nennt er die Bedeutung des Naturbegriffs durch explemplarische Aufzählungen. 
\begin{quote}
	\glqq Unter den vorhandenen (Dingen) sind die einen von Natur aus, die anderen sind auf Grund anderer Ursachen da. Von
Natur aus: Die Tiere und deren Teile, die Pflanzen und die einfachen unter den Körpern, wie Erde, Feuer, Luft und
Wasser; von diesen und Ähnlichem sagen wir ja, es sei von Natur aus. \grqq{}
	\cite[192b8 ff.]{PhysikII} 
\end{quote}
Mit diesem Gedanken zeigt Aristoteles was er unter Natur versteht. Er beschreibt den Begriff der Natur so, wie er heute noch seine Gültigkeit hat. Genauso wie die Pflanzen zur Natur gehören, gehören auch die Tiere dazu. Beide gehen mit einer Bewegung einher. Sei es das Wachstum einer Pflanze oder die Bewegung eines Tieres. Um die Frage von oben nochmals aufzugreifen: Besitzen die Dinge aus der Natur ein Bewusstsein? Speziell auf die zwei Beispiele bezogen würden wir bei einem sofort zustimmen. Bei Tieren sind wir der Meinung, dass diese durchaus ein Bewusstsein über sich selbst haben. Ein Wolf weiß zum Beispiel, dass er ein Wolf ist und an welcher Stelle in der Nahrungskette er steht. Wir sind auch der Überzeugung, dass Pfanzen eine Art Bewusstsein haben. Da wir bisher das Bewusstsein nicht definieren können, kann es eine Art Bewusstsein sein, die uns völlig fremd ist. Pflanzen haben zwar kein Gehirn das unserem ähnelt. Aber der Aufbau einer Pflanze im Inneren kann als Nervenbahnen verstanden werden. Da wir derzeit die Funktionsweise unseres Gehirns noch nicht entschüsselt haben, können wir auch nicht sagen, ob eine Pflanze eine Art Gehirn in sich trägt. Es gibt Wissenschafter auf diesem Planeten die durchaus überzeugt sind, dass die Pflanzen ein Bewusstsein haben\footnote{ vgl. \url{https://www.welt.de/wissenschaft/article5804911/Pflanzen-besitzen-eine-besondere-Intelligenz.html}}.    
Weiter zu Dingen aus der Natur heißt es:
\begin{quote}
	\glqq Von diesen hat nämlich ein jedes in sich selbst einen Anfang von Veränderung und Bestand, teils bezogen auf Raum, teils auf Wachstum und Schwinden, teils auf Eigenschaftsveränderung. \grqq{}
	\cite[192b8 ff.]{PhysikII} 
\end{quote}
Daraus lassen sich abstrakt wesentliche Merkmale von \glqq Naturdingen\grqq{} ableiten. So unterliegen natürliche Gegenstände entweder ganz oder zu teilen der Veränderung. Zum Anderen, tragen natürliche Gegenstände ihre Bewegung und Veränderung in sich selbst. Sie werden also nicht von außen angeleitet etwas zu tun.

Nun nennt Aristoteles Dinge die nicht zur Natur gehören:
\begin{quote}
	\glqq Hingegen, Liege und Kleid, und was es dergleichen Gattungen sonst noch geben mag, hat, insofern ihm eine jede solche Bezeichnung eignet und insoweit es ein kunstmäßig hergestelltes Ding ist, keinerlei innewohnenden Drang zu Veränderung in sich; \grqq{}
	\cite[192b8 ff.]{PhysikII} 
\end{quote}
Aristoteles benennt nun die Dinge, die nicht zur Natur gehören, als kunstmäßig hergestelltes Ding. Dieses Ding besitzt, nachdem es hergestellt wurde, keinen Drang mehr zur Veränderung. Technische Dinge (kunstmäßig hergestellte Dinge) existieren nicht von Natur aus, sondern sind entstanden aufgrund von anderen Ursachen. Erst der Erschaffer eines technischen Dings verleiht dem Ding Funktion und Form. Bezogen auf einen Roboter: Ein Roboter besteht nicht von Natur aus. Er hat einen Schöpfer, der sich Gedanken über die Funktion und Form des Roboter macht. Der Roboter an für sich hat keine Bestrebung sich zu verändern oder zu bewegen. Erst die Funktion des Roboters erzeugt eine Bewegung oder Veränderung. Somit ist ein Roboter aus Aristoteles Sicht ein technisches Ding. 

Um die Brücke zur \ac{ki} zu schlagen: Laut der Definition von Aristoteles müsste eine \ac{ki} ein technisches Ding sein,da es ein kunstmäßig hergestelltes Ding ist. Es gab irgendwann mal einen oder mehrere Schöpfer. Allerdings unterliegt es dem Drang der Veränderung. Angenommen wir hätten eine starke \ac{ki}, diese wird versuchen sich möglichst viel Wissen anzueignen, ohne das jemand von außen einen Impuls dazu gibt. Was dem Drang einer Veränderung gleich kommt. Um ein konkretes Beispiel zu nennen. Angenommen die \ac{ki} hat ein künstliches neuronales Netzwerk. Dieses Netzwerk scheint wie Menschen zu lernen und zu verstehen. In der Tat sollen sie ein menschliches Gehirn simulieren. Durch das \glqq Lernen\grqq{} passieren im Netzwerk diverse Änderungen, was mit Drang zur Veränderung/Bewegung einhergeht. Also warum sollte eine starke \ac{ki} nicht der Natur zugeordnet werden? Unsere Aussage begann mit \glqq Es gab irgendwann mal einen oder mehrere Schöper\grqq{}, wer sagt denn, dass wir keinen Schöpfer haben? Dann wären wir auch nur ein kunstmäßig hergestelltes Ding?


