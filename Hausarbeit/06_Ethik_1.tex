\section{Ethiken}
In diesem Kapitel möchten wir darauf eingehen wie verschiedene Ethiker den Einsatz von Chatbots bewerten könnten. Dazu sei gesagt, das es dazu nicht die richtige Antwort gibt. Wir versuchen unsere Aussagen anhand von Gedanken der Ethiker zu belegen. 

\subsection{Aristoteles}
Im ersten Abschnitt wird kurz auf die Person Aristoteles eingegangen werden. \newline
Aristoteles zählt bis heute noch zu den bekanntesten und einflussreichsten Philosophen und Naturforschern. Geboren 384 v. Chr. in Stagira (Griechenland) und gestorben 322 v. Chr. in Chalkis. Bereits zu dieser Zeit setzte sich Aristoteles mit zahlreichen Themengebieten auseinander. Kurz gesagt war er Wissenschaftler, Biologe, Physiker und Philosoph. Zu den berühmtesten Werken des Aristoteles zählen seine Poetik, Politik und Metaphysik.\footnote{vgl. \url{https://de.wikipedia.org/wiki/Aristoteles}}



An dieser Stelle sollen ein paar Gedanken aus dem Dialog zwischen Harald Lesch und Wilhelm Vossenkuhl aufgenommen werden. In diesem Dialog unterhalten sich die beiden über Aristoteles. Dabei arbeiten die beiden bestimmte Thesen aus, die dann Stelle zur eigenen Meinungsfindung verwendet werden. Harald Lesch ist Physiker an der Ludwig-Maximilians-Universität München. Neben der Physik beschäftigt er sich mit der Philosophie. Wilhelm Vossenkuhl ist ein emeritierter Professor für Philosophie auch an der Ludwig-Maximilians-Universität in München. Dieser Dialog ist als Video verfügbar in der ARD Mediathek\footnote{vgl. \url{http://www.ardmediathek.de/tv/Denker-des-Abendlandes/Aristoteles/ARD-alpha/Video?bcastId=14913016&documentId=15666426}}. Hiermit sei darauf hingewiesen, dass die folgenden Gedanken nicht von den Autoren stammen. Die Gedanken der zwei Diaglogführenden werden in den eigenen Worten der Autoren wiedergeben. Es werden nur die Aussagen herangezogen die dann später zur Meinungsfindung verwendet werden.


Aristoteles ist am Anfang der Metaphysik der Überzeugung, dass es das natürliche Bestreben des Menschen ist zu Wissen. Wissen ist etwas was neu werden kann, es erweitert sich. Weiter erwähnen die zwei Protagonisten, dass Aristoteles der erste große Logiker war. Die Logik ist eine Art Handwerkszeug für die Lösung von Problemen. Welches Werkzeug aus meinem Werkzeugkoffer muss ich verwenden, um ein bestehendes Problem zu lösen. Die Frage kommt auf \glqq Wie komme ich (Harald Lesch) eigentlich zu irgendwelchen Schlüssen über Phänomene?\grqq. Lesch bezieht dies speziell auf die Astrophysik. Die Antwort liegt nahe, man verwendet die Logik um diese Phanömen handhabar zu machen. Aristoteles hat die Syllogistik (Logik) so weit entwickelt, dass für jede denkbare theoretische Situation ein Schlussverfahren möglich ist. Das Konzept besteht dabei aus Obersatz, Untersatz und Schluss. Eine Anmerkung seitens der Autoren, dieses Konzept kommt auch in der Rechtslehre zur Anwendung. Die Protagonisten bewegen sich nun weg von der Logik. Aristoteles hatte die Vorstellung davon, dass alles wegen einem gewissen Zweck geschieht. Aus der Nikomachische Ethik des Arsistoteles geht hervor, dass jedes Problem seine ihm einge Genauigkeit hat. Das heißt einmal ist es gut genau hinzuschauen und einmal erweist es sich als besser weniger genau hinzuschauen. Kurz gesagt man soll nicht immer alles mit dem gleichen Maßstab messen. Weiter erkennt Aristoteles, dass man mit der Syllogistik keine Ethik treiben kann. Somit ist klar, dass man ethische Fragen nicht mit wahr-falsch beantworten kann. Allgemein trennt Aristoteles die Methoden der Wissenschaft und Ethik. Für die Wissenschaft gibt es die Syllogistik. Bei der Ethik kommt die Fuzzylogik (Aussage Harald Lesch) ins Spiel. Abschließend kommt der Entschluss, die Gesamtheit einer ethischen Frage kann man nicht richtig greifen. 

Nun möchten wir ein paar Gedanken des Dialogs zu unserer eigenen Meinungsfindung verwenden. Ist es aus Sicht von Aristoteles vertretbar einen Chatbot zu verwenden? Jeder Mensch hat das Bestreben zu Wissen laut Aristoteles. Er nennt hier aber in keiner Form wie die Übermittlung des Wissens stattfindet. Auch nennt er keine Quelle für das Wissen. Unser \glqq Problem\grqq\ ist es, dass das Wissen was der Mensch erfährt von einer Maschine/\ac{ki} stammt. Wir sind der Meinung, dass die Maschine als Wissenquelle in ein paar Jahren nicht mehr zur Debatte steht. Wissenquellen haben sich seit anbeginn der Zeit verändert. Zuerst waren es die Gelehrten die das Wissen vermittelten. Oft nur durch mündliche Weitergabe. Denn der Rest der Bevölkerung konnte nicht lesen und somit aufgeschriebenes Wissen nicht konsumieren. Also oblag die Wissenweitergabe den Gelehrten. Mit dem Fortschreiten der Bildung unter der \glqq normalen\grqq\ Bevölkerung, war es auch für diese möglich niedergeschriebenes Wissen zu konsumieren. Nun kommen neben Gelehrten auch Aufzeichungen (Bücher, etc.) als Wissenquelle ins Spiel. Die Menschen können sich nun unabhängig von den Gelehrten Wissen aneigen und weitergeben. Die Wissenquelle Aufzeichnungen steht heute für uns außer Frage. Nach den Bücher kommt für uns das Internet als Wissenquelle hinzu. Beim Internet als Wissenquelle kann man schon diskutieren. Aber wenn man weiß wie man das Internet als Wissenquelle zu verwenden hat, ist diese Wissenquelle genau so geeignet wie Bücher. Die Autoren stellen das Internet als Wissenquelle nicht in Frage. Wir sind der Überzeugung, dass dies auch mit der \ac{ki} als Wissenquelle im Laufe der Zeit passieren wird. Als Conclusio können wir sagen, dass eine \ac{ki} das Bestreben des Menschen nach Wissen nicht im Wege steht, eher sogar fordert. \newline
Aristoteles hat sich wie eingangs erwähnt viele Gebieten gewidmet. Unter anderem auch der Wissenschaft. Aristoteles ist bekannt für seine Syllogistik. Eine abgewandelte Form dieser Syllogistik wird auch in der \ac{ki} verwendet. Was wir heute Aussagenlogik nennen kommt der Vorstellung von Aristoteles am nächsten. Er trennt zwar die Wissenschaft mit ihrer Method der Logik von der Ethik, allerdings spielt uns die Tatsache, dass Aristoteles sich mit der Logik beschäftigt hat in die Karten. Da die Logik ein Baustein der \ac{ki} ist wird Aristoteles nicht abgeneigt von der Idee einer \ac{ki} sein. \newline
Auch erkannte Aristoteles, dass ethische Fragen nicht pauschal mit wahr oder falsch beantwortet können. Daran hat sich auch heute noch nichts geändert. Wir sind heute noch in dem Dilema ethische Fragen zum Wohle aller beteiligten zu beantworten. Lesch beschreibt dieses Problem mit dem Ausdruck Fuzzylogik. Die Fuzzylogik zeichnet sich durch ihre unschärfe in der Formulierung aus. Es kann nicht mehr auf wahr oder falsch geschlossen werden. Es ist irgendwas zwischen diesen Zuständen. \newline
Wie die zwei Dialogführenden weiter herausfiltern hat jedes Problem seine eigene Genauigkeit. In unserem Fall ist es nicht getan das Thema nur kurz zu betrachten. Auch Aristoteles wäre der Ansicht, dass dieses Thema einer genaueren Untersuchung bedarf.

Abschließend bleibt uns zu sagen, dass wir sehr erstaunt über die Einsichten von Aristoteles sind. Seine strikte Trennung von Wissenschaft und Ethik ist heute noch gültig. Auf der Grundlage der bisher diskutierten Gedanken sehen wir keine Handhabe die gegen die \ac{ki} spricht. 


