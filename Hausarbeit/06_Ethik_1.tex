\section{Ethiken}
In diesem Kapitel möchten wir darauf eingehen wie verschiedene Ethiker den Einsatz von Chatbots bewerten könnten. Dazu sei gesagt, dass es dazu nicht die richtige Antwort gibt. Wir versuchen unsere Aussagen anhand von Gedanken der Ethiker zu belegen. 


\subsection{Platon}
Platon war ein griechischer Philosoph, der auf die gesamte Entwicklung der Philosophie einen großen Einfluss hatte. Er war Schüler des Sokrates und Lehrer des Aristoteles, der ihm jedoch in zentralen Fragen widersprach. Geboren wurde Platon 427/428 \ac{vCHR} in Athen, wo er auch 80 Jahre später, im Jahre 347 \ac{vCHR}, verstarb. In den Gebieten der objektiv-idealistischen Philosophie, der Metaphysik, der Erkenntnistheorie, der Ethik, der Anthropologie, der Staatstheorie, der Kosmologie, der Kunsttheorie und der Sprachphilosophie war er richtungsweisend. Der Mittelpunkt seiner Philosophie bildet die Ideenlehre.\footnote{vgl. \url{http://www.whoswho.de/bio/platon.html}}

\subsubsection{Die Ideenlehre}
Ideen sind Intelligibel, nicht an einen Körper gebunden.
\subsubsection{Meinungsfindung}
Passt schon!






