\section{Einleitung}
Die \ac{ki} ist einer der zentralen Wegbereiter für den technologischen Fortschritt der Menschheit im 21. Jahrhundert.
Gestützt durch jährlich exponentielles Wachstum der Rechenleistung von Computern, stößt sie in immer weitere Bereiche des menschlichen Lebens vor.
Durch mediale Präsenz bekannte Beispiele hierfür sind die semantische Suchmaschine Watson\footnote{Die von IBM entwickelte \ac{ki} schaffte es im Jahr 2011 bei der Quizsendung \textit{Jeopardy!} gegen zwei Menschliche Konkurrenten zu gewinnen.} oder autonom fahrende Automobile. %QUELLEN ANGEBEN
Doch auch für die Produktionssteigerung in der Industrie, für verbesserte Behandlungsmöglichkeiten in der Medizin und für viele weitere Gebiete ist sie entscheidend.
Es ist abzusehen, dass durch anhaltende Verbesserungen an \ac{ki}-Systemen in den nächsten Jahrzehnten eine massive Revolution hinsichtlich unserer Lebensweise bevorsteht.

Doch wie die meisten technologischen Fortschritte hat auch die \ac{ki} ihre Schattenseiten.
Was passiert, wenn es uns gelingt eine künstliche Intelligenz zu schaffen, die der Intelligenz des Menschen ebenbürtig ist und sich anschließend aus eigener Kraft rasant selbst verbessern kann?
Dieser Zeitpunkt des menschlichen Fortschritts wird \textbf{technologische Singularität} genannt.
%TODO weitere Negativbeispiele, Bezug zu Ethik + Thema ersichtlich machen, Quellen + Bilder..

\subsection{Motivation}
Die Motivation uns in dieser Hausarbeit mit dem Thema \enquote{künstliche Intelligenz} auf Basis verschiedener ethischer Standpunkte kritisch auseinander zusetzen, fußt auf ihrer großen Relevanz für unser bereits jetziges und vor allen Dingen zukünftiges Leben.
%TODO besser begründen

\subsection{Fragestellung}
Bei dem Thema \textit{Künstliche Intelligenz -- Warnung vor der Singularität} stellt sich die Frage wie der Einfluss der künstlichen Intelligenz auf unsere Gesellschaft und unser Leben von verschiedenen ethischen Standpunkten aus gesehen zu bewerten sind.


\subsection{Problemstellung}
Durch den technischen Fortschritt der \ac{ki} sind eine Vielzahl an Problemen zu erwarten.
Diese Hausarbeit beschäftigt sich explizit mit folgenden für die Ethik relevanten Problemstellungen: 
\begin{itemize} 
	\item Direkte Gefahr für den Menschen durch die \ac{ki} (militärischer Einsatz).
	\item Arbeitslosigkeit als Resultat der \ac{ki} verursachten Vollautomatisierung.   
	\item Gleichberechtigung von Mensch und \ac{ki}.
	\item Werte und Rechtsverständnis einer \ac{ki}.
\end{itemize}
%TODO Anpassen nachdem wir besser recherchiert haben