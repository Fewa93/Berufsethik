\section{Einleitung}
Die \ac{ki} ist einer der zentralen Wegbereiter für den technologischen Fortschritt der Menschheit im 21. Jahrhundert.
Gestützt durch jährlich exponentielles Wachstum der Rechenleistung von Computern, stößt sie in immer weitere Bereiche des menschlichen Lebens vor.
Durch mediale Präsenz bekannt gewordene Beispiele hierfür sind die semantische Suchmaschine Watson\footnote{Die von IBM entwickelte \ac{ki} schaffte es im Jahr 2011 bei der Quizsendung \textit{Jeopardy!} gegen zwei Menschliche Konkurrenten zu gewinnen.} oder autonom fahrende Automobile. %QUELLEN ANGEBEN
Doch auch für die Produktionssteigerung in der Industrie, für verbesserte Behandlungsmöglichkeiten in der Medizin und für viele weitere Gebiete ist sie entscheidend.
Es ist abzusehen, dass durch anhaltende Verbesserungen an \ac{ki}-Systemen in den nächsten Jahrzehnten eine massive Revolution hinsichtlich unserer Lebensweise bevorsteht.

Doch wie die meisten technologischen Fortschritte hat auch die \ac{ki} ihre Schattenseiten.
Was passiert, wenn es uns gelingt eine künstliche Intelligenz zu schaffen, die der Intelligenz des Menschen ebenbürtig ist und sich anschließend aus eigener Kraft rasant selbst verbessern kann?
Dieser Zeitpunkt des menschlichen Fortschritts wird \textbf{technologische Singularität} genannt.

Für uns Menschen ist es, spätestens ab diesem Zeitpunkt, wichtig zu wissen, wie eine \ac{ki} in ethisch fragwürdigen Situationen reagiert und entscheidet. Aber auch bei aktuellen Systemen die auf \ac{ki} basieren, wie beispielsweise autonom fahrende Autos oder Chatbots mit \ac{ki}, taucht diese Frage auf. 
%TODO drüber schauen, ergänzen, verbessern

\subsection{Motivation}
Die Motivation uns in dieser Hausarbeit mit dem Thema \enquote{künstliche Intelligenz} und deren ethische Vertretbarkeit auseinander zu setzen hat vielerlei Gründe. 
Vorrangig besitzt dieses Thema große Relevanz für unser jetziges und vor allen Dingen zukünftiges Leben. 
Zusätzlich ist die \ac{ki} für uns Master-Studenten des Bereichs Informatik ein spannendes und interessantes Gebiet, in dem wir während unseres Bachelorstudiums schon erste Eindrücke und Erfahrungen sammeln durften.
Die Verknüpfung mit dem Bereich Ethik entstand erstmals durch Vorträge unseres Professors für \ac{ki} Prof. Dr. rer. nat. Wolfgang Ertel\footnote{\cite{ProfessorErtel}}. Endgültig entschiedenen wir uns für die Kombination dieser beiden Gebiete durch die Wahl unseres Studienschwerpunktes IT-Sicherheit, in dem diese Vorlesung eine Pflichtveranstaltung ist.
%TODO drüber schauen, ergänzen, verbessern

\subsection{Themenabgrenzung}
Da die Gebiete der Ethik und der \ac{ki} zusammen den Rahmen dieser Hausarbeit sprengen würden, ist es notwendig das Thema auf einen Bruchteil zu begrenzen. Im Bereich der \ac{ki} konzentrieren wir uns deshalb auf den Einsatz von Chatbots mit künstlicher Intelligenz. Der ethische Teil wird durch die drei bekannten Ethiker Platon, Aristoteles und Nietzsche repräsentiert. 
%TODO drüber schauen, ergänzen, verbessern

\subsection{Fragestellung}
\textbf{Ist der Einsatz von Chatbots mit \ac{ki} ethisch vertretbar?}
Dieser Frage gehen wir in der Hausarbeit nach, in dem wir den Sachverhalt auf die Ansichten verschiedener Ethiker abbilden und versuchen daraus ein Fazit zu ziehen.
%TODO drüber schauen, ergänzen, verbessern
