\section{Einleitung}
Die \ac{ki} ist einer der zentralen Wegbereiter für den technologischen Fortschritt der Menschheit im 21. Jahrhundert.
Gestützt durch jährlich exponentielles Wachstum der Rechenleistung von Computern stößt sie in immer weitere Bereiche des menschlichen Lebens vor.
Durch mediale Präsenz bekannte Beispiele hierfür sind die semantische Suchmaschine Watson\footnote{Die von IBM entwickelte \ac{ki} schaffte es im Jahr 2011 bei der Quizsendung \textit{Jeopardy!} gegen zwei Menschliche Konkurrenten zu gewinnen.} oder autonom fahrende Automobile. %QUELLEN ANGEBEN
Doch auch für die Produktionssteigerung in der Industrie, für verbesserte Behandlungsmöglichkeiten in der Medizin und für viele weitere Gebiete ist sie entscheidend.
Es ist abzusehen, dass durch anhaltende Verbesserungen an \ac{ki}-Systemen in den nächsten Jahrzehnten eine massive Revolution hinsichtlich unserer Lebensweise bevorsteht.

Doch wie die meisten technologischen Fortschritte hat auch die \ac{ki} ihre Schattenseiten.
Was passiert, wenn es uns gelingt eine künstliche Intelligenz zu schaffen, die der Intelligenz des Menschen ebenbürtig ist und sich anschließend aus eigener Kraft rasant selbst verbessern kann?
Dieser Zeitpunkt des menschlichen Fortschritts wird \textbf{technologische Singularität} genannt.
%TODO weitere Negativbeispiele, Bezug zu Ethik + Thema ersichtlich machen, Quellen + Bilder..

\subsection{Motivation}




\subsection{Fragestellung}


\subsection{Problemstellung}