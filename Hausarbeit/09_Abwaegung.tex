\section{Abwägung}

Nachdem wir drei Ethiker vorgestellt haben, möchten wir in diesem Kapitel die drei Ansichten gegeneinander betrachten. Wie das vorherige Kapitel gezeigt hat, gibt es sowohl Zusammenhänge als auch Wiedersprüche in deren Gedankenwelt.

Beginnen möchten wir mit der These von Platon \glqq Wahrnehmung ist ungleich Wissen\grqq. Wie bereits beschrieben, ist Platon der Meinung, dass wir durch unsere Sinne getäuscht werden. Die Wahrnehmung ist nicht gleich Wissen. Übertragen auf unseren Chatbot bedeutet dies, dass die Informationen des Chatbots kein Wissen darstellen. \newline	
Gegenüber steht die These von Aristoteles \glqq Das natürliche Bestreben des Menschen ist zu Wissen\grqq. Er ist der Ansicht, dass Wissen etwas ist, was sich erweitert. Unsere Ausarbeitung führt uns dazu, dass Chatbots das Bestreben nach Wissen befriedigen. Denn der Chatbot liefert uns eine Antwort auf eine Frage. Dabei war uns die Antwort der Frage natürlich vorher unbekannt, sonst würden wir nicht fragen. Unser Wissen hat sich nun erweitert. Wir wissen nun zum Beispiel die Nennspannung eines elektrischen Gerätes. Wie bereits bei der Vorstellung der Ethiker erwähnt, waren sich Platon und Aristoteles in ihrer Weltanschaung nicht einig. Diese erste Differenz sehen wir am oben genannten Beispiel. \newline
Nun versuchen wir nietzsche These \glqq Die Ewige Wiederkunft\grqq\ mit einzubringen. Dabei beschreibt er das Universum als ein zyklisches System, indem sich alle Zustände bereits unendlich of wiederholt haben. Bedeutet dies nicht, dass wir dann in einer Schleife gefangen sind? 



