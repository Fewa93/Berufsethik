\section{Abwägung}
\textbf{1. Abwägung:}

In diesem Kapitel sollen ein paar grundsätzliche Fragen in bezug auf Chatbots und Ethik diskutiert werden. 
Als Grundlage dient uns das Dokument \glqq Entscheidungsunterstützung mit Künstlicher Intelligenz\grqq\footnote{vgl. \cite{Bitkom}}, verfasst von Bitcom und dem Deutsches Forschungszentrum für Künstliche Intelligenz GmbH. 
Speziell Kapital 8 \glqq Automatisierte Entscheidungen aus ethischer Sicht\grqq\ wird für unsere Abwägung herangezogen. 
Wir begeben uns nun in die Sicht eines Anwenders, der mit einem Chatbot kommuniziert. 

Wie wir es bereits heute erleben, werden immer mehr Systeme \glqq intelligent\grqq. 
Sei es ein Chatbot, der mit einer \ac{ki} arbeitet oder ein autonom fahrendes Auto. 
Hinter diesen Prozessen steckt der Gedanke der Prozessoptimierung und der effizienteren Gestaltung von Prozessen. 
Doch wir dürfen an dieser Stelle die Ethik nicht vergessen. 
Was für den einen Menschen von Vorteil sein mag, das stellt sich für andere Menschen womöglich als Nachteil heraus.

\textbf{Chanchengleichheit} ist hier der Punkt. Wie kann sichergestellt werden, dass durch die \ac{ki} im Hintergrund keine Diskriminierung stattfindet. Sei es aufgrund des Geschlechts, der ethnischen Herkunft, der Religion oder der sexuellen Überzeugung. Auch heute noch ist die Homosexualität ein Tabuthema in weiten Teilen der Gesellschaft. 
Der Rechtsstaat hat zwar die Ehe zwischen homosexuellen Paaren erlaubt, allerdings bedeutet dies nicht, dass Homosexuelle dadurch automatisch in der Gesellschaft anerkannt werden. 
Es gibt nach wie vor viele Vorurteile gegenüber dieser sexuellen Orientierung. 
Bei der Auswahl für eine freie Arbeitsstelle könnte beispielsweise ein homosexueller gegenüber heterosexuellen Bewerbern im Nachteil sein.

Was ist, wenn dem Chatbot die sexuelle Orientierung bekannt ist und er aus diversen Quellen gelernt hat, dass Homosexualität nicht gut ist? 
Dies führt zu der oben genannten Chancenungleichheit. 
%Wahrscheinlich ist es dem Anwender nicht einmal bewusst, dass der Chatbot eine Art Vorurteil gegenüber dem Anwender hat.

\textbf{Informationsfreiheit und freie Meinungsbildung} ist der nächste Punkt. Dies umfasst zunächst mal den Zugang zu Informationen.
Wie kann der Zugriff auf Informationen gewährleistet werden? Was ist wenn Chatbots Falschmeldungen verteilen? Wie können die Bürger davor geschützt werden?

Auch Prof. Dr. Oliver Bendel\footnote{Er ist Ethiker und Wirtschaftsinformatiker an der Fachhochschule Nordwestschweiz.} sieht hier eine große Gefahr. 
Es gibt bereits Newsportale, die absichtlich Lügen verbreiten -- Stichwort \enquote{Fakenews}. 
Diese tragen dann indirekt zur Meinungsbildung bei. 
Die Erstellung der Falschmeldungen geschieht zum Teil durch Menschen als auch durch Maschinen. 

Zu Demonstrationszwecken entwickelte er den Lügenbot. Dieser Chatbot ist konzipiert dazu Lügen zu verbreiten. 
Das Ziel dieses Projekts ist es, die Strategien des maschinellen Lügens aufzuzeigen und zu verstehen. 
Dies soll dabei helfen, Programmierern sowie Anwendern mögliche Gefahren, die durch diese Technik entstehen kann aufzuzeigen.\footnote{vgl. \cite{Bendel}} 

PLATON

Wie bereits Aristoteles erkannte hat jedes Problem seine eigene Genauigkeit. Speziell Chatbots und die \ac{ki} bedürfen einer sehr hohen Genauigkeit. Wir müssen uns im klaren sein, welche Auswirkungen unsere technologischen Fortschritte auf uns Menschen haben. Verschwindet vielleicht die menschliche Komponente durch den Einsatz eines Chatbots? 
Es gibt unzählige Fragen, die genau betrachtet und bewertet werden müssen. 
Das menschliche Bestreben nach Wissen kann ein Chatbot möglicherweise abdecken, jedoch muss weiterhin sichergestellt werden, dass die zur Verfügung gestellten Informationen auch richtig sowie frei von Vorurteilen sind.

NIETZSCHE

%ABWÄGUNG IST WAAAAAAAAAAS?
%Eine Abwägung stellt in der Rechtswissenschaft die Ergebnisse von zwei oder mehreren zu entscheidenden Fragestellungen in ein Verhältnis, das die sich aus den Fragestellungen ergebende Entscheidung als möglichst gerecht darstellt. Auch außerhalb der Rechtswissenschaft stellt eine Abwägung die Vorbereitung einer Entscheidung dar, bei der die absehbaren Folgen einer Entscheidung ermittelt und zu den Zielen, zu denen die Entscheidung führen soll, ins Verhältnis gesetzt werden. Zwischen vielen Faktoren gibt es Trade-offs; siehe auch Kosten-Nutzen-Abwägung, Kompromiss und Zielkonflikt.%

%Die Abwägung durchläuft drei Phasen:

%Zusammenstellung des Abwägungsmaterials
%Bewertung der Einzelbelange
%Vorgang des untereinander und gegeneinander Abwägens der Belange
%

\textbf{2. Abwägung:}


In der Abwägung greifen wir die Konklusionen der einzelnen Ethiker auf und setzen diese in Bezug auf unsere heutige Weltanschauung. Auch werden die Anwender des Chatbots mit in Betracht gezogen.

\textbf{Platon}

Unsere Konklusion von Platons möglicher Ansicht zielt auf die Einordnung und Qualität des Wissens beziehungsweise der Informationen des Chatbots ab. Auch die Verwechslungsgefahr und Irreführung durch einen Chatbot mit \ac{ki} sind ein Teil dieser möglichen Ansicht.

Aus heutiger Sicht haben wir bereits das Problem, dass falsch Aussagen durch das Internet verbreitet werden. Auch Prof. Dr. Oliver Bendel\footnote{Er ist Ethiker und Wirtschaftsinformatiker an der Fachhochschule Nordwestschweiz.} sieht hier eine große Gefahr. 
Es gibt bereits Newsportale, die absichtlich Lügen verbreiten -- Stichwort \enquote{Fakenews}. 
Diese tragen dann indirekt zur Meinungsbildung bei. 
Die Erstellung der Falschmeldungen geschieht zum Teil durch Menschen als auch durch Maschinen. 

Zu Demonstrationszwecken entwickelte er den Lügenbot. Dieser Chatbot ist konzipiert dazu Lügen zu verbreiten. 
Das Ziel dieses Projekts ist es, die Strategien des maschinellen Lügens aufzuzeigen und zu verstehen. 
Dies soll dabei helfen, Programmierern sowie Anwendern mögliche Gefahren, die durch diese Technik entstehen kann aufzuzeigen.\footnote{vgl. \cite{Bendel}}  Die Sorge, dass falsche Informationen von einem Chatbot verbreitet werden und der Benutzer diese als Wahr aufnimmt, ist definitiv vorhanden.

Die zweite Kernthese, dass der Chatbot mit einem Menschen verwechselt werden könnte, ist zum Teil in der heutigen Zeit schon real.
Dies zeigt das Ergebnis des Turing-Tests der University of Reading beim Chatbot \glqq Eugene Goostman\grqq\, der es schaffte 33 \% der 30 Testteilnehmer zu täuschen.\footnote{vgl. \cite{UnivOfReading}} 

Diese Beispiele zeigen die enorme Bedeutung von Platons Ansichten für unsere Fragestellung. 

\textbf{Aristoteles}
Unsere Konklusion von Aristoteles möglicher Ansicht zielt auf die Förderung der Wissenserlangung und der differenzierten Betrachtungsweise eines jeden Themas.

Wissen stellt in unserer Gesellschaft eine sehr wichtige Ressource dar. Es heißt nicht umsonst \glqq Wissen ist Macht\grqq. Allerdings leben wir heute in einer Zeit in der Wissen im Überfluss vorhanden ist. Das heißt wir müssen uns primär nicht darum kümmern wie wir an Wissen gelangen, sondern das benötigte Wissen zu identifizieren und herauszufiltern.
Ein Chatbot könnte für den Benutzer eine Quelle für gefiltertes Wissen darstellen und somit Aristoteles Ansicht der Wissenserlangung unterstützen.

Eine differenzierte Betrachtung der Frage, ob Chatbots mit \ac{ki} ethisch vertretbar sind, wie es Aristoteles machen würde, ist auch für uns von hoher Bedeutung. Es müssen alle Blickwinkel betrachtet werden um ein möglichst genaues Fazit ziehen zu können.
 
Die aufgeführten Beispiele zeigen eine, dass auch Aristoteles Aussagen für unsere Fragestellung durchaus von Bedeutung sind. 

%ALTER TEXT
\textbf{Aristoteles}

Wie aus der Konklusion von Aristoteles hervorgeht, hat der Mensch das Bestreben zu Wissen und jedes Problem hat seine eigene Genauigkeit.

Wissen stellt in unserer Gesellschaft eine sehr wichtige Ressource dar. Es heißt nicht umsonst \glqq Wissen ist Macht\grqq. Allerdings leben wir heute in einer Zeit in der Wissen im Überfluss vorhanden sind. Das heißt wir müssen uns nicht mehr darum kümmern wie wir an Wissen gelangen. Wir haben eher das Problem das notwendige bzw. richtige Wissen zu identifizieren. Des Weiteren müssen wir die Wissensquellen hinterfragen. Wie bereits genannt, gibt es auch Quellen die absichtlich falsches Wissen verteilen. Auch Platon erkannte dies und sagt, dass das Wissen eine gewisse Qualität haben muss. Ansonsten bringt uns das Wissen nicht weiter und wir stehen wir am Anfang. Wir sind der Überzeugung, dass Aristoteles dies genau gleich sehen würde.  

Sind Chatbots nun ethisch vertretbar? Nun wir können auch nicht die eine richtige Antwort geben. Die Antwort bedarf einer genaueren Analyse. Wir können das sehr gut an dieser Hausarbeit sehen. Auch hier wird die Frage von mehreren Blickwinkeln betrachtet und bewertet. Um das auf Aristoteles Ansicht zu beziehen, wir legen in diesem Fall eine hohe Genauigkeit an. Wir haben erkannt, dass die Frage nicht einfach zu beantworten ist. 

% 1) Konklusion Zusammengefasst%

%  2) 1 unter Berücksichtigugn der heutigen Zeit/Erkenntnisse/...

% 3) Bedeutung von 2 für unsere Fragestellung

\textbf{Nietzsche}

In unserer Konklusion zu Nietzsche haben wir festgehalten, dass dieser möglicherweise den Chatbot in Teilen als eine Form des Übermenschen betrachten könnte, dessen Entwicklung unbedingt durch den Menschen vorangetrieben werden müsste.

Selbstverständlich ist der direkte Vergleich zwischen Chatbot und Übermensch gewagt.
Zum einen wurde das Konstrukt des Übermenschen darauf ausgelegt, sich auf eine überlegende Spezies zu beziehen, die sich direkt aus dem Menschen durch eine Art Evolution heraus entwickelt hat. 
Die zukünftigen technischen Errungenschaften im Bereich der \ac{ki} wurden dabei im 18. Jahrhundert bei seiner Definition sicherlich noch nicht bedacht.   
Zum anderen ist es aus heutiger Sicht noch völlig unklar wohin und wie weit sich die dem Chatbot zugrundeliegende \ac{ki}-Forschung überhaupt entwickeln kann und somit auch, ob überhaupt jemals eine Intelligenz von Menschenhand geschaffen werden kann, welche die Attribute des Übermenschen hinreichen erfüllen kann.

So sehr der Vergleich zwischen Chatbot und Übermensch allerdings auch hinken mag, so stecken dennoch zutreffende Gedanken in ihm.
Sicherlich sind viele Menschen von dem Gedanken fasziniert eine Art Gott zu spielen und etwas zu schaffen, das ihnen selbst überlegen ist.
Es ist daher auch davon auszugehen, dass sich viele Wissenschaftler nicht mit dem bloßen Bestehen des Turing-Tests zufrieden geben werden, sondern wohl doch mit dem Ziel forschen eine ihnen überlegene Intelligenz -- \enquote{den Übermenschen} -- zu erschaffen.



