\section{Abwägung}

Nachdem im vorherigen Kapitel die Standpunkte der Philosophen Platon, Aristoteles sowie Nietzsche zu Chatbots betrachtet wurden, soll nun anhand der Meinung der Autoren eine Abwägung getroffen werden, welcher philosophischer Standpunkt dieser drei am ehesten vertretbar ist.

Als zweifellos größter Befürworter von Chatbots ist Nietzsche zu sehen.
Nichts menschengemachte kommt seinem Konstrukt des Übermenschen näher als die künstliche Intelligenz eines Chatbots.
Auch wir, als Befürworter von Chatbots, halten diesen Standpunkt für zutreffend.
So lässt der exponentiellen Fortschritt der Leistungsfähigkeit ihrer zugrundeliegenden Intelligenz die Annahme zu, dass noch zu unseren Lebzeiten mit der Schaffung einer uns überlegenen Intelligenz zu rechnen ist.
Sollte dies gelingen, so ist auch unserer Ansicht nach der Begriff Übermensch absolut zutreffend.