\section{Abwägung}

In der Abwägung greifen wir die Konklusionen der einzelnen Ethiker auf und setzen diese in Bezug auf unsere heutige Weltanschauung. Auch werden die Anwender des Chatbots mit in Betracht gezogen.

\textbf{Platon}

Unsere Konklusion von Platons möglicher Ansicht zielt auf die Einordnung und Qualität des Wissens beziehungsweise der Informationen des Chatbots ab. Auch die Verwechslungsgefahr und Irreführung durch einen Chatbot mit \ac{ki} sind ein Teil dieser möglichen Ansicht.

Aus heutiger Sicht haben wir bereits das Problem, dass Falschaussagen durch das Internet verbreitet werden. Auch Prof. Dr. Oliver Bendel\footnote{Er ist Ethiker und Wirtschaftsinformatiker an der Fachhochschule Nordwestschweiz.} sieht hier eine große Gefahr. 
Es gibt bereits Newsportale, die absichtlich Lügen verbreiten -- Stichwort \enquote{Fakenews}. 
Diese tragen dann indirekt zur Meinungsbildung bei. 
Die Erstellung der Falschmeldungen geschieht zum Teil durch Menschen als auch durch Maschinen. 

Zu Demonstrationszwecken entwickelte er den Lügenbot. Dieser Chatbot ist konzipiert dazu Lügen zu verbreiten. 
Das Ziel dieses Projekts ist es, die Strategien des maschinellen Lügens aufzuzeigen und zu verstehen. 
Dies soll dabei helfen, Programmierern sowie Anwendern mögliche Gefahren, die durch diese Technik entstehen kann aufzuzeigen.\footnote{vgl. \cite{Bendel}}  Die Sorge, dass falsche Informationen von einem Chatbot verbreitet werden und der Benutzer diese als Wahr aufnimmt, ist definitiv vorhanden.

Die zweite Kernthese, dass der Chatbot mit einem Menschen verwechselt werden könnte, ist zum Teil in der heutigen Zeit schon real.
Dies zeigt das Ergebnis des Turing-Tests der University of Reading beim Chatbot \glqq Eugene Goostman\grqq, der es schaffte 33 \% der 30 Testteilnehmer zu täuschen.\footnote{vgl. \cite{UnivOfReading}} 

Diese Beispiele zeigen die enorme Bedeutung von Platons Ansichten für unsere Fragestellung. 

\textbf{Aristoteles}

Unsere Konklusion von Aristoteles möglicher Ansicht zielt auf die Förderung der Wissenserlangung und der differenzierten Betrachtungsweise eines jeden Themas.

Wissen stellt in unserer Gesellschaft eine sehr wichtige Ressource dar. Es heißt nicht umsonst \glqq Wissen ist Macht\grqq. Allerdings leben wir heute in einer Zeit, in der Wissen im Überfluss vorhanden ist. 
Das heißt wir müssen uns primär nicht darum kümmern wie wir an Wissen gelangen, sondern wie wir das benötigte Wissen identifizieren und herausfiltern.
Ein Chatbot könnte für den Benutzer eine Quelle für gefiltertes Wissen darstellen und somit Aristoteles Ansicht der Wissenserlangung unterstützen.

Eine differenzierte Betrachtung der Frage, ob Chatbots mit \ac{ki} ethisch vertretbar sind, wie es Aristoteles machen würde, ist auch für uns von hoher Bedeutung. Es müssen alle Blickwinkel betrachtet werden um ein möglichst genaues Fazit ziehen zu können.
 
Die aufgeführten Beispiele zeigen, dass auch Aristoteles Aussagen für unsere Fragestellung durchaus von Bedeutung sind. 

\textbf{Nietzsche}

In unserer Konklusion zu Nietzsche haben wir festgehalten, dass dieser möglicherweise den Chatbot in Teilen als eine Form des Übermenschen betrachten könnte, dessen Entwicklung unbedingt durch den Menschen vorangetrieben werden müsste.

Selbstverständlich ist der direkte Vergleich zwischen Chatbot und Übermensch gewagt.
Zum einen wurde das Konstrukt des Übermenschen darauf ausgelegt, sich auf eine überlegende Spezies zu beziehen, die sich direkt aus dem Menschen durch eine Art Evolution heraus entwickelt hat.
Die zukünftigen technischen Errungenschaften im Bereich der \ac{ki} wurden dabei im 18. Jahrhundert bei seiner Definition sicherlich noch nicht bedacht.   
Dadurch gelingt es nicht wirklich den Chatbot in das Definitionsraster des Übermenschen hineinzuzwängen.

Zum anderen ist es aus heutiger Sicht noch völlig unklar wohin und wie weit sich die dem Chatbot zugrundeliegende \ac{ki}-Forschung überhaupt entwickeln kann und somit auch, ob überhaupt jemals eine Intelligenz von Menschenhand geschaffen werden kann, welche die Attribute des Übermenschen hinreichen erfüllen kann.

Aufgrund dieser Probleme sind für die Klärung unserer Frage, ob die Benutzung von Chatbots ethisch vertretbar ist, beim derzeitigen Stand der Technik die Gedankenspiele Nietzsches weitestgehend irrelevant.


\textbf{Zusammenfassung}

Mit Nietzsches Ansätzen sehen wir wenige parallelen zu unserer Fragestellung und schließend diesen somit aus unserem Fazit aus. Wie wir herausgearbeitet haben, sind die Ansichten von Platon und Aristoteles für unsere Fragestellung allerdings von Bedeutung. 

Wie in Kapitel 2 erwähnt, widersprachen sich Platon und Aristoteles in zentralen Fragen. Dies spiegeln unsere Konklusionen der beiden Ethiker wieder. Platon, der sich vermutlich eher gegen die Verwendung eines Chatbots mit \ac{ki} aussprechen würde und Aristoteles, der ihn wohl befürworten würde. Beide haben mit ihren Thesen für unsere Fragestellung eine große Relevanz und wir können an dieser Stelle nicht sagen, dass der eine deutlich zutreffendere Referenzen zu unserer Fragestellung hat als der andere. 

Dies ist nicht weiter schlimm, denn obwohl die Konklusionen von Platon und Aristoteles zu einer unterschiedlichen Beantwortung unserer Frage führten, beharren sie auf anderen Betrachtungsweisen des Themas. Aus diesem Grund werden wir im Fazit die Ansichten dieser beiden Ethiker berücksichtigen und sie nicht gegeneinander ausspielen. Die Fuzzylogik spiegelt diese Schlussfolgerung wieder, denn eine ethische Frage kann nicht mit wahr oder falsch beantwortet werden sondern liegt irgendwo dazwischen. 



