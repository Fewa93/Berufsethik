\section{Grundlagen}
Dieses Kaptiel soll die Grundlagen für das Verständis dieser Hausarbeit nahelegen. Die ersten zwei Abschnitte definieren die \ac{ki} und die möglichen Auswirkungen. Die nächsten Abschnitte sollen die \glqq Erfinder\grqq{} der jeweiligen Ethik vorstellen. 


\subsection{Künstliche Intelligenz}
Dieser Abschnitt definiert die Bedeutung der \ac{ki} aus Sicht der Autoren. Bereits viele Menschen, vor den Autoren, haben versucht die \ac{ki} zu definieren. Leider gibt es nicht die eindeutige Definition der \ac{ki}. Bereits bei der Frage \glqq Was ist Intelligenz\grqq{} gibt es nicht die einzig wahre Aussage. Was sicher ist, die Menschen nehmen eine besondere Stellung unter den Lebewesen ein. Diese besondere Stellung basiert auf unserer Intelligenz. \linebreak 
Hierzu ein kleines Beispiel, ist ein Roboter intelligent, wenn er einem Hinderniss, zum Beispiel einer Kiste, ausweichen kann?  Viele bezeichnen dieses Verhalten des Roboters bereits als intelligent. Aus Sicht von Informatikern ist dieses Verhalten nicht als intelligent zu bezeichen. Vielmehr ist eine logische Schlussfolgerung aus eingehenden Sensorsignalen. Anhand von Sensoren, kann ein Roboter seine Umgebung wahrnehmen. Bekommt der Roboter nun den Input, dass vor ihm ein Hinderniss ist, dann ändert dieser, aufgrund der Programmierung, seine Richtung. Wenn ein Mensch einem Hinderniss ausweicht, wird das nicht unbedingt als Intelligenzleistung angesehen. Sicherlich ist es intelligent einem Hinderniss auszuweichen, um physische Einwirkungen auf den menschlichen Körper zu vermeinden. Allerdings stellt dies keine Aufgabe im Sinne einer Intelligenzleistung dar. Laut Duden wird die Intelligenz definiert als \glqq Fähigkeit des Menschen, abstrakt und vernünftig zu denken und daraus zweckvolles Handeln abzuleiten\grqq. In diesem Fall ist das Ausweichen des Roboters zwar zweckvoll, aber stellen wir uns mal vor, dass neben der Kiste links und rechts ein tiefer Abgrund ist. Hinter der Kiste führt ein sicherer Weg weiter. Der Mensch würde über die Kiste steigen und den sicheren Weg nehmen. Auch wenn der Mensch vorher noch nie einen tiefen Abgrund hintergefallen ist, weiß er, dass ein Ausweichen der Kiste die falsche Entscheidung darstellt. Der Roboter  wird der Kiste ausweichen und vermutlich in den Abgrund stürzen. Dies ist dann durchaus kein intelligentes Handeln mehr. \linebreak 
Der \ac{ki} Pionier John McCarthy \cite{PROPOSALMcCarthy} veröffentlichte bereits 1955 eine Exposé in der McCarthy auf die Künstliche Intelligenz eingeht. Die Exposé definiert die \ac{ki} wie folgt 
\begin{quote}
		For the present purpose the artificial intelligence problem is taken to be that of making a machine behave in ways that would be called intelligent if a human were so behaving.
\end{quote}
Das bedeutet soviel wie, dass Maschinen sich verhalten sollen, als verfügten sie über Intelligenz. Mit Bezug auf des vorangegange Beispiel würde unser Roboter laut dieser Definition als Intelligent eingestuft werden. \linebreak
Es gibt noch diverse andere Definition der \ac{ki}. Eine der wohl besten Definitionen lieferte Elanie Rich 1983 \cite{ArtificialIntelligence}. 
\begin{quote}
		Artificial Intelligence is the study of how to make computers do things at which, at the moment, people are better.
\end{quote}
Im speichern von Daten und der Berechnung von nummerischen Aufgaben sind Computer uns um ein Vielfaches überlegen. 
Wie sich die Aufgabenbreiche der \ac{ki} verändern kann man an zwei Beispielen sehen. Im Jahr 1997 schlägt IBM's Deep Blue den Schach Weltmeister Garri Kasparow\footnote{vgl. \url{http://de.chessbase.com/post/20-jahre-kasparov-gegen-deep-blue}}. Dies hatte zur Folge, dass die \ac{ki} nach und nach das Interesse an Schach verlor. Heute gilt Schach als gelöst aus Sicht der \ac{ki}. Anfang 2016 gab es einen weitere Sensation Alpha Go von Google schlägt menschlichen Go-Profi\footnote{vgl. \url{https://www.heise.de/newsticker/meldung/Google-KI-schlaegt-menschlichen-Profi-Spieler-im-Go-3085855.html}}. Auch diese Tatsache wird dazu führen, dass sich die \ac{ki} auf neue Aufgabenbereiche konzentriert. Es gibt allerdings noch zahlreiche Gebiete in denen wir Menschen der \ac{ki} weit überlegen sind. Für uns ist es zum Beispiel kein Problem einen Raum zu betreten und den Raum durch eine andere Tür wieder zu verlassen. Der Mensch kann diese Situation binnen Sekundenbruchteilen deuten und falls nötig eine Entscheidung treffen. Allein schon das Erkennen einer Tür in einem Raum ist für die \ac{ki} nicht trivial. Folglich stellt das Verlassen eines Raumes für einen Robotor eine derzeit nicht lösbare Aufgabe dar. \linebreak
Allgemein kann man sagen, dass die menschliche Intelligenz bei der Adaptivität liegt. Das beudeutet wird sind in der Lage uns an verschiedene Situationen anzupassen und daraus zu lernen.   

\subsection{Technologische Singularität}
Unter der technologischen Singularität wird der Zeitpunkt verstanden, an dem die Maschinen intelligenter sind als wir Menschen. Niemand kann de facto heute vorrausagen ob die technologische Singularität jemals eintreten wird. Einige Science-Fiction Werke wie Matrix oder Terminator zeigen bereits eine Zukunft in der die Maschinen uns überlegen sind. In beiden Beispielen ist das Ziel der Maschinen die Menschheit vernichten, weil die Maschinen eine Bedrohung für die eigene Existens in den Menschen sehen. Um diese Entscheidung zu treffen, muss die \ac{ki} allerdings ihr eigenes Bewusstsein entwickeln. Die \ac{ki} muss im Stande sein, über sich selbst zu reflektieren. Diese Form wird auch Starke \ac{ki} genannt, die \ac{ki} besitzt die gleichen intellektuellen Fertigkeiten wie der Mensch. Dazu gehören folgende Fähigkeiten:
\begin{itemize}
	\item Logisches Denken
	\item Treffen von Entscheidungen bei Unsicherheit
	\item Planen      
	\item Lernen
	\item Kommunikation in einer Sprache
	\item Bewusstsein
	\item Eigenwahrnehmung
	\item Empfindungsvermögen
\end{itemize}
Was wir heute in der \ac{ki} vorfinden ist die Form der schwachen \ac{ki}. Diese zeichnet sich durch Systeme aus, die spezielle Aufgaben wie Schach lösen. Alles was über Schach hinaus geht, ist für das System nicht lösbar. \linebreak
Ein kleines Beispiel soll aufzeigen was ein Mensch im Hintergrund alles verarbeitet bei der Nennung einer Zeichenfolge:
\begin{quote}
	\quad 10/2017 
\end{quote}
Diese Zeichenfolge ist die Aneinaderreihung von Zeichen aus einem Zeichenvorrat. Der Mensch assoziert sofort Okotober 2017 mit dieser Zeichenfolge. Vielleicht kommt einem noch mehr in den Sinn, was man anhand von Erfahrungen mit dem Okotober verknüpft. Wenn man an das Auto denkt, man sollte so langsam an die Winterreifen denken. Für die \ac{ki} ist es eine Zeichenfolge wie jede Andere. Erst wenn die \ac{ki} gelernt hat, dass es sich dabei um ein Datum handelt, können weitere Interpretationen folgen. Dabei tritt das erste Problem schon auf, was ist, wenn sich die Schreibweise der Zeichenfolge ändert aber die Bedeutung gleich bleibt.
\begin{quote}
	\quad 	10.2017	\quad 	10\_2017	 \quad 10|2017  
\end{quote}
Sicherlich kann man der \ac{ki} alle Formen eines Datums beibringen, allerdings folgt das nächste Problem gleich darauf. Nehmen wir folgende Aussage:
\begin{quote}
	\quad Haltbarkeitsdatum Joghurt: 10/2057 
\end{quote}
Was den Menschen zum Schmunzeln bringt, da ein Joghurt niemals bis 2057 haltbar sein kann, wenn er heute produizert wurde. Für uns kann es nur ein Fehler sein. Die \ac{ki} kann dies allerdings ohne, dass man es ihr gelehrt hat, nicht erkennen. Wir Menschen wissen auch nicht das genaue Haltbarkeitsdatum, aber aufgrund unserer Erfahrung mit Milchprodukten können wir die Haltbarkeit abschätzen. So können wir auch ohne weiteres die Haltbarkeit von Quark abschätzen oder jede andere Form von Milchprodukten. Von solchen Gedanken/Interpretationen ist die heutige \ac{ki} noch weit entfernt. 

\subsection{Ethik-Typ-1}

\subsection{Ethik-Typ-2}

\subsection{Ethik-Typ-3}
