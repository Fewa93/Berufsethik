\section{Grundlagen}
Dieses Kapitel soll die Grundlagen für das Verständnis dieser Hausarbeit nahelegen. Die ersten zwei Abschnitte definieren die \ac{ki} und Chatbots. 

\subsection{Künstliche Intelligenz}
Dieser Abschnitt definiert die Bedeutung der \ac{ki} aus Sicht der Autoren. Bereits viele Menschen haben sich daran versucht den Begriff der \ac{ki} zu definieren. Leider gibt es bislang keine allgemein anerkannte und eindeutige Definition. Bereits bei der Frage \enquote{Was ist Intelligenz} gibt es nicht eine einzig wahre Aussage. Sicher ist, die Menschen nehmen eine besondere Stellung unter den Lebewesen ein. Diese besondere Stellung basiert unter anderem auf unserer Intelligenz. 

Der \ac{ki} Pionier John McCarthy veröffentlichte bereits 1955 eine Exposé in der McCarthy auf die Künstliche Intelligenz eingeht. Die Exposé definiert die \ac{ki} wie folgt:
\begin{quote}
		\glqq For the present purpose the artificial intelligence problem is taken to be that of making a machine behave in ways that would be called intelligent if a human were so behaving.\grqq\footnote{\cite{PROPOSALMcCarthy}}
\end{quote}
Das bedeutet soviel wie, dass Maschinen sich verhalten sollen, als verfügten sie über Intelligenz. Wir persönlich finden diese Aussage zu vage. Den für viele Menschen gilt ein Roboter, der einem Hinderniss ausweicht schon als intelligent. Für uns Informatiker ist das Ausweichen eine logische Schlussfolgerung aus eingehenden Sensorsignalen. Bekommt der Roboter die Sensoreingabe, dass er vor einem Hindernis steht, so ändert dieser aufgrund der Programmierung die Richtung.

Einen weiteren Versuch die \ac{ki} zu definieren, unternahm Elanie Rich bereits 1983: 
\begin{quote}
	 \glqq Artificial Intelligence is the study of how to make computers do things at which, at the moment, people are better.\grqq\footnote{\cite{ArtificialIntelligence}}
\end{quote}

Kurz gesagt bedeutet dies, dass die \ac{ki} Dinge ausführen oder erledigen soll, in denen die Menschen im Moment noch besser sind. Dazu zwei Beispiele: \newline
Im Speichern von Daten und der Berechnung von nummerischen Aufgaben sind Computer uns um ein Vielfaches überlegen. \newline
 Die Menschen sind allerdings in der Erkennung von Objekten den aktuellen Algorithmen weit überlegen. Sobald wir einen Raum betreten findet im unserem Unterbewusstsein eine Objekterkennung statt. Wir erkennen sofort, dass der Raum beispielsweise drei Fenster, zwei Türen und vier Wände hat. Gleichzeit erkennen wir Gegenstände im Raum, wie Tische, Stühle und Bildschirme auf den Tischen. Danach schließen wir darauf, dass dieser Raum ein Computerraum sein muss. Dieser Entschluss wird gefasst mit Hilfe von Wissen was wir bereits haben und Erfahrungen die wir erlebt haben. Wir verknüpfen innerhalb von Sekunden die Objekte und unser vorhandenes Wissen, um einen Entschluss zu fassen. Die aktuelle \ac{ki} steckt hier noch in den Kinderschuhen. Moderne Algorithmen können zwar mehr oder weniger gut Objekte erkennen und diese Greifen, aber die \ac{ki} hat kein Gesamtbild der Umgebung. Die zwei Beispiele sollen zeigen, dass es Dinge gibt die ein Computer aktuell besser kann aber auch Dinge die ein Computer noch nicht besser kann als ein Mensch. Die KI ist ständig im Wandel. Gilt ein Problem als gelöst dann verschieben sich die Aufgabenbereiche der KI.   
Wie sich die Aufgabenbereiche der \ac{ki} verändern zeigen zwei Beispiele. Im Jahr 1997 schlägt IBM's Deep Blue den Schach Weltmeister Garri Kasparow\footnote{vgl. \cite{SchachQuelle}}. Dies hatte zur Folge, dass die \ac{ki}-Forschung nach und nach das Interesse an Schach verlor. Aus ihrer Sicht gilt Schach heute als gelöst. 

Anfang 2016 gab es einen weitere Sensation. Die \ac{ki} Alpha Go von Google schlägt einen menschlichen Go-Profi\footnote{vgl. \cite{GoQuelle}}. Auch diese Entwicklung führt dazu, dass sich die \ac{ki}-Forschung auf neue Aufgabenbereiche konzentrieren wird. 

Es gibt noch zahlreiche Gebiete in denen wir Menschen der \ac{ki} weit überlegen sind. Durch die rasante Entwicklung der letzten Jahre in der \ac{ki} werden immer mehr Anwendungen und Produkte mit ihr verknüpft. So setzen Firmen bereits sogenannte Chatbots, mit einer \ac{ki} im Hintergrund, zur Kundenkommunikation ein. Was ein Chatbot ist und das die Idee nicht neu ist soll der nächste Punkt zeigen. 

\subsection{Chatbot}
Ein Chatbot ist eine Art Maschine für die Kommunikation mit dem Menschen. Meistens besitzt ein Chatbot ein Dialogsystem. Das heißt der Kommunikationspartner kann per Texteingaben mit dem Chatbot kommunizieren. Der Mensch stellt eine Frage per Texteingabe. Der Chatbot versucht daraufhin die Frage zu interpretieren und generiert eine Antwort. Die Maschine muss man sich als reine Software vorstellen. Sie besitzt keine materielle Erscheinung, lediglich das System auf dem die Software ausgeführt wird, ist Materiell.

Die Idee eine Maschine zur Kommunikation mit dem Menschen einzusetzen ist nicht neu. Bereits 1966 entwickelte Joseph Weizenbaum ein Computerprogramm, dass die Kommunikation mit Mensch und Computer ermöglichte. Das Programm wird \glqq ELIZA\grqq\ genannt. Im Hintergrund verwendet das Programm ein sogenanntes \glqq Pattern Matching\grqq\, was einer Mustererkennung entspricht. Mit dieser eigentlich einfachen Technik, war es möglich den ersten Chatbot zu programmieren. An dieser Stelle soll nicht weiter auf technische Details eingegangen werden.\footnote{vgl. \cite{WikiELIZA}}    

Heute setzen immer mehr Firmen Chatbots in der Kundenkommunikation ein, speziell beim Support von Kunden. So greifen Firmen wie Lufthansa, Zalando, Opel, etc. bereits auf Chatbots zurück. Die Chatbots sind zwar noch nicht voll integriert, aber erste Experimente finden bereits statt.\footnote{vgl. \cite{UnternehmenChatbots}} Getrieben durch die Fortschritte in der \ac{ki} werden die Chatbots immer besser. Vor allem das maschinelle Lernen hilft den Chatbots sich kontinuierlich zu verbessern. Es wird immer schwieriger den Gegenüber als Chatbot zu identifizieren. Die University of Reading führte 2014 den sogenannten Turing-Test\footnote{Der Bericht definiert den Test als bestanden: Wenn ein Chatbot für mehr als 5 Minuten für einen Menschen gehalten wird und wenn mehr 30 \% der Testteilnehmer getäuscht werden} beim Chatbot \glqq Eugene Goostman\grqq\ durch. Der Chatbot schaffte es 33 \% der 30 Testteilnehmer zu täuschen.\footnote{vgl. \cite{UnivOfReading}}

Bitkom untersuchte mit Hilfe einer Umfrage\footnote{vgl. \cite{BitkomChatbot}} den Einsatz von Chatbots unter den Bundesbürgern. Die Umfrage wurde am 18.01.2017 mit dem Titel \glqq Jeder Vierte will Chatbots nuten\grqq\ veröffentlicht. Es wurden im November 2016 insgesamt 1.005 Personen ab 14 Jahren in Deutschland befragt. Die Umfrage kam zu interessanten Ergebnissen:
\begin{itemize}
 	\item Die Umfrage ergab, dass 63 Prozent keine Chatbots nutzen wollen. Das heißt sie möchten nicht mit einer Maschine kommunizieren. 
	 \item Das Anfragen zuverlässig bearbeitet werden können bezweifeln etwa 50 Prozent.
	 \item Das Chatbots uninteressant sind, weil die \ac{ki} noch nicht ausgereift ist, denken 47 Prozent.
\end{itemize}
Dieser Auszug aus der Studie zeigt uns, dass die Bundesbürger zum Zeitpunkt der Studie eher skeptisch zum Thema Chatbots sind. Es besteht allgemein noch eine Ablehnung gegenüber Chatbots. Wie sich dieses Gefüge mit der Zeit verschieben wird kann keiner sagen.  

TODO: VLLT NOCH WEITER AUSSCHREIBEN MIT EIGENER MEINUNG DER AUTOREN?







