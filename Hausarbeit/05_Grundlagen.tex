\section{Grundlagen}
Dieses Kaptiel soll die Grundlagen für das Verständis dieser Hausarbeit nahelegen. Die ersten zwei Abschnitte definieren die \ac{ki} und die möglichen Auswirkungen. Die nächsten Abschnitte sollen die \enquote{Erfinder} der jeweiligen Ethik vorstellen. 

\subsection{Künstliche Intelligenz}
Dieser Abschnitt definiert die Bedeutung der \ac{ki} aus Sicht der Autoren. Bereits viele Menschen haben sich daran versucht den Begriff der \ac{ki} zu definieren. Leider gibt es bislang keine allgemein anerkannte und eindeutige Definition für sie. Bereits bei der Frage \enquote{Was ist Intelligenz} gibt es nicht eine einzig wahre Aussage. Sicher ist, die Menschen nehmen eine besondere Stellung unter den Lebewesen ein. Diese besondere Stellung basiert auf unserer Intelligenz.

Hierzu ein kleines Beispiel: Ist ein Roboter intelligent, wenn er einem Hindernis, beispielsweise einer Kiste, ausweichen kann?  Viele würden dieses Verhalten des Roboters bereits als intelligent bezeichnen. Aus Sicht von Informatikern ist dieses Verhalten allerdings nicht aufgrund einer Intelligenz gegeben. Vielmehr ist es eine logische Schlussfolgerung aus eingehenden Sensorsignalen. Anhand von Sensoren kann ein Roboter seine Umgebung wahrnehmen. Bekommt der Roboter nun die Sensoreingabe, dass er vor einem Hindernis steht, so ändert dieser aufgrund der Programmierung die Richtung. 

Wenn ein Mensch einem Hindernis ausweicht, wird das nicht unbedingt als Intelligenzleistung angesehen. Sicherlich ist es intelligent einem Hindernis auszuweichen, um physische Einwirkungen auf den menschlichen Körper zu vermeiden. Allerdings stellt dies keine Aufgabe im Sinne einer Intelligenzleistung dar. Laut Duden wird die Intelligenz definiert als \enquote{Fähigkeit des Menschen, abstrakt und vernünftig zu denken und daraus zweckvolles Handeln abzuleiten}. %TODO QUELLE
In diesem Fall ist das Ausweichen des Roboters zwar zweckvoll, aber stellen wir uns mal vor, dass sich neben der Kiste links und rechts ein tiefer Abgrund befände. Hinter der Kiste führt ein sicherer Weg weiter. Der Mensch würde über die Kiste steigen und den sicheren Weg nehmen. Auch wenn der Mensch vorher noch nie einen tiefen Abgrund hinuntergefallen ist, so weiß er, dass ein Ausweichen der Kiste die falsche Entscheidung darstellt. Der Roboter  wird der Kiste ausweichen und vermutlich in den Abgrund stürzen. Dies ist dann durchaus kein intelligentes Handeln mehr. 

Der \ac{ki} Pionier John McCarthy \cite{PROPOSALMcCarthy} veröffentlichte bereits 1955 eine Exposé in der McCarthy auf die Künstliche Intelligenz eingeht. Die Exposé definiert die \ac{ki} wie folgt:
\begin{quote}
		For the present purpose the artificial intelligence problem is taken to be that of making a machine behave in ways that would be called intelligent if a human were so behaving.
\end{quote}
Das bedeutet soviel wie, dass Maschinen sich verhalten sollen, als verfügten sie über Intelligenz. Mit Bezug auf des vorangegangene Beispiel würde unser Roboter laut dieser Definition als Intelligent eingestuft werden.

Es gibt noch diverse andere Definition der \ac{ki}. Eine der wohl besten Definitionen lieferte Elanie Rich 1983 \cite{ArtificialIntelligence}: 
\begin{quote}
		Artificial Intelligence is the study of how to make computers do things at which, at the moment, people are better.
\end{quote}

Im Speichern von Daten und der Berechnung von nummerischen Aufgaben sind Computer uns um ein Vielfaches überlegen. 
Wie sich die Aufgabenbereiche der \ac{ki} verändern zeigen zwei Beispiele. Im Jahr 1997 schlägt IBM's Deep Blue den Schach Weltmeister Garri Kasparow\footnote{vgl. \url{http://de.chessbase.com/post/20-jahre-kasparov-gegen-deep-blue}}. Dies hatte zur Folge, dass die \ac{ki}-Forschung nach und nach das Interesse an Schach verlor. Aus ihrer Sicht gilt Schach heute als gelöst. 

Anfang 2016 gab es einen weitere Sensation. Die \ac{ki} Alpha Go von Google schlägt einen menschlichen Go-Profi\footnote{vgl. \url{https://www.heise.de/newsticker/meldung/Google-KI-schlaegt-menschlichen-Profi-Spieler-im-Go-3085855.html}}. Auch diese Entwicklung führt dazu, dass sich die \ac{ki}-Forschung auf neue Aufgabenbereiche konzentrieren wird. 

Es gibt allerdings noch zahlreiche Gebiete in denen wir Menschen der \ac{ki} weit überlegen sind. Für uns ist es zum Beispiel kein Problem einen Raum zu betreten und den Raum durch eine andere Tür wieder zu verlassen. Der Mensch kann diese Situation binnen Sekundenbruchteilen deuten und falls nötig eine Entscheidung treffen. Allein schon das Erkennen einer Tür in einem Raum ist für die \ac{ki} nicht trivial. Folglich stellt das Verlassen eines Raumes für einen Robotor eine derzeit nicht lösbare Aufgabe dar. \linebreak
Allgemein kann man sagen, dass die menschliche Intelligenz bei ihrer Adaptivität liegt. Das bedeutet, dass wir in der Lage sind uns an verschiedene Situationen anzupassen und daraus zu lernen.   

\subsection{Technologische Singularität}
Unter der technologischen Singularität wird der Zeitpunkt verstanden, an dem die Maschinen intelligenter sind als wir Menschen. Niemand kann de facto heute voraussagen, ob die technologische Singularität jemals eintreten wird. 

Einige Science-Fiction Werke wie Matrix oder Terminator zeigen bereits eine Zukunft, in der die Maschinen uns überlegen sind. In beiden Beispielen ist das Ziel der Maschinen die Menschheit vernichten, weil die Maschinen eine Bedrohung für die eigene Existenz in den Menschen sehen. Um diese Entscheidung zu treffen, muss die \ac{ki} allerdings ein eigenes Bewusstsein entwickeln. Die \ac{ki} muss im Stande sein, über sich selbst zu reflektieren. Diese Form wird auch starke \ac{ki} genannt -- die \ac{ki} besitzt die gleichen intellektuellen Fertigkeiten wie der Mensch. Dazu gehören folgende Fähigkeiten:
\begin{itemize}
	\item Logisches Denken
	\item Treffen von Entscheidungen bei Unsicherheit
	\item Planen      
	\item Lernen
	\item Kommunikation in einer Sprache
	\item Bewusstsein
	\item Eigenwahrnehmung
	\item Empfindungsvermögen
\end{itemize}
Was wir heute in der \ac{ki} vorfinden, wird als schwache \ac{ki} bezeichnet. Diese äußert sich in Systemen zum Lösen spezieller Aufgaben, wie beispielsweise Schach. Alles was über Schach hinaus geht, ist für das System nicht lösbar.

Ein kleines Beispiel soll aufzeigen, was ein Mensch im Hintergrund alles verarbeitet bei der Nennung einer Zeichenfolge:
\begin{quote}
	\quad 10/2017 
\end{quote}
Diese Zeichenfolge ist die Aneinanderreihung von Zeichen aus einem Zeichenvorrat. Der Mensch assoziiert sofort Oktober 2017 mit dieser Zeichenfolge. Vielleicht kommt einem noch mehr in den Sinn, was man anhand von Erfahrungen mit dem Oktober verknüpft. Wenn man beispielsweise an das Auto denkt, so sollte man allmählich an die Winterreifen denken. 
Für die \ac{ki} ist es eine Zeichenfolge wie jede andere. Erst wenn die \ac{ki} gelernt hat, dass es sich dabei um ein Datum handelt, können weitere Interpretationen folgen. 

Dabei tritt das erste Problem schon auf. Was passiert, wenn sich die Schreibweise der Zeichenfolge ändert aber die Bedeutung gleich bleibt.
\begin{quote}
	\quad 	10.2017	\quad 	10\_2017	 \quad 10|2017  
\end{quote}
Sicherlich kann man der \ac{ki} alle Formen eines Datums beibringen, allerdings folgt das nächste Problem gleich darauf. Nehmen wir folgende Aussage:
\begin{quote}
	\quad Haltbarkeitsdatum Joghurt: 10/2057 
\end{quote}
Für uns Menschen kann es nur ein Fehler sein, da ein Joghurt niemals bis 2057 haltbar sein kann, wenn er kürzlich produziert wurde.
Eine \ac{ki} kann dies allerdings ohne weiteres nicht plausibilisieren. Wir Menschen kennen auch nicht das genaue Haltbarkeitsdatum, können aber aufgrund unserer Erfahrung mit Milchprodukten die ungefähre Haltbarkeit abschätzen. So können wir auch ohne weiteres die Haltbarkeit von Quark abschätzen oder jeder anderen Form von Milchprodukten. Von solchen tiefgreifenden Schlussfolgerungen  ist die heutige \ac{ki} noch weit entfernt. 

\subsection{Ethiken}
Die folgenden Abschnitte sollen die Ethiken vorstellen, die in dieser Hausarbeit berücksichtigt werden. 
Da ihre Autoren zahlreiche Publikationen veröffentlicht haben, werden in dieser Hausarbeit nur die wichtigsten und zum Thema passenden Werke berücksichtigt.
%TODO ausarbeiten

\subsubsection{Aristoteles}
Aristoteles zählt bis heute noch zu den bekanntesten und einflussreichsten Philosophen und Naturforschern. Geboren 384 v. Chr. in Stageira (Griechenland) und gestorben 322 v. Chr. in Chalkis. Bereits zu dieser Zeit setzte sich Aristoteles mit zahlreichen Themengebieten auseinander. Kurz gesagt war er Wissenschaftler, Biologe, Physiker und Philosoph. Zu den berühmtesten Werken des Aristoteles zählen seine Poetik, Politik und Metaphysik.

\subsubsection{Ethik-Typ-2}

\subsubsection{Ethik-Typ-3}
