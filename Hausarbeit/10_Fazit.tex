\section{Fazit}
Unserer Ansicht nach ist es nur unter bestimmten Voraussetzungen ethisch Vertretbar einen Chatbot mit zugrundeliegender \ac{ki} zu verwenden.
Zunächst müsste gewährleistet werden, dass keine falschen oder irreführenden Informationen mithilfe des Chatbots verbreitet werden. Dadurch wären auch die möglichen Bedenken Platons hinfällig. 

Chatbots müssen auch allgemeinen ethischen Grundsätzen folgen. Ein negativ Beispiel liefert uns der Chatbot \glqq Tay\grqq\ von Microsoft. Dieser wurde innerhalb von Stunden zum Rassisten. Wie war dies möglich? Der Chatbot war so konzipiert, dass er aus Benutzereingaben lernte. Das Problem war, dass Tay alles ungefiltert aufnahm was die Benutzer eingaben. So lernte er auch von menschenverachtenden Eingaben. Da sich die Menschen einen Spaß daraus machten Tay mit solchen Texten zu \glqq füttern\grqq, war sein Schicksal besiegelt. Er lernte den Inhalt und gab ihn wieder. So wurde er zum Rassisten. Microsoft musste den Chatbot nach nicht einmal einem Tag abschalten.\footnote{vgl. \cite{TaySpiegel}} 

Es darf nicht nur ein paar Individuen obliegen intelligente Systeme zu implementieren. Ferner muss eine Gesellschaft die Handlungsweise des Systems definieren und überwachen. 
Aristoteles, der die Förderung der Wissensgewinnung und die Logik der \ac{ki} wahrscheinlich befürworten würde wäre somit wahrscheinlich auch der Ansicht, dass der Einsatz von Chatbots ethisch vertretbar ist.

Abschließen möchten wir diese Arbeit mit einem Zitat\footnote{Dieses Zitat stammt aus der Publikation \glqq Entscheidungsunterstützung mit Künstlicher Intelligenz\grqq. Sie befasst sich unter anderem mit automatisierten Entscheidungen aus ethischer Sicht.}:
\begin{quote}
	 \glqq Eine menschengerechte Einbindung intelligenter Systeme in hochkomplexe Gesellschaften ist keine individuelle Angelegenheit, sondern eine gesellschaftliche Aufgabe.\grqq\footnote{\cite{BitkomZitat}}
\end{quote}
