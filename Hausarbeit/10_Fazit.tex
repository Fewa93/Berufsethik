\section{Fazit}
Unserer Ansicht nach ist es nur unter bestimmten Voraussetzungen ethisch Vertretbar, einen Chatbots mit \ac{ki} zu verwenden.\newline
Zunächst müsste gewährleistet werden, dass keine falschen oder irreführenden Informationen über den Chatbot verbreitet werden. Dann wären auch die möglichen Bedenken Platons hinfällig. \newline
Chatbots müssen auch allgemeinen ethischen Grundsätzen folgen. Ein negativ Beispiel liefert uns der Chatbot \glqq Tay\grqq\ von Microsoft. Dieser wurde innerhalb von Stunden zum Rassisten. Doch wie war dies möglich? Der Chatbot war so konzipiert, dass er aus den Benutzereingaben lernte. Das Problem war, dass Tay alles ungefiltert aufnahm was die Benutzer eingaben. So lernte er auch von menschenverachtenden Eingaben. Da sich die Menschen einen Spaß daraus machten Tay mit solchen Texten zu \glqq füttern\grqq, war sein Schicksal besiegelt. Er lernte den Inhalt und gab ihn wieder. So wurde er zum Rassisten. Microsoft musste den Chatbot nach nicht einmal einem Tag abschalten.\footnote{\cite{TaySpiegel}} \newline
Es darf nicht ein paar Induvidien obliegen, intelligente Systeme zu implentieren. Ferner muss eine Gesellschaft die Handlungsweise des Systems definieren und überwachen. Allerdings liefert uns die Vergangenheit bereits ein negativ Beispiel. \newline
Aristoteles der die Förderung der Wissensgewinnung und die Logik der \ac{ki} wahrscheinlich befürworten würde wäre somit auch der Ansicht, das der Einsatz von Chatbot ethisch vertretbar ist.

Abschließen möchten wir unsere Hausarbeit mit einem Zitat beenden. Dieses Zitat stammt aus der Publikation \glqq Entscheidungsunterstützung mit Künstlicher Intelligenz\grqq\footnote{vgl. \cite{Bitkom}}. Diese Publikation befasst sich unter anderem mit automatisierten Entscheidungen aus ethischer Sicht.
\begin{quote}
	 \glqq Eine menschengerechte Einbindung intelligenter Systeme in hochkomplexe Gesellschaften ist keine individuelle Angelegenheit, sondern eine gesellschaftliche Aufgabe.\grqq\footnote{\cite{BitkomZitat}}
\end{quote}
