\section{Fazit}
Unserer Ansicht nach ist es nur unter bestimmten Voraussetzungen ethisch Vertretbar, einen Chatbots mit \ac{ki} zu verwenden.
Zunächst müsste gewährleistet werden, dass keine falschen oder irreführenden Informationen über den Chatbot verbreitet werden. Dann wären auch die möglichen Bedenken Platons hinfällig. Aristoteles der die Förderung der Wissensgewinnung und die Logik der \ac{ki} wahrscheinlich befürworten würde wäre somit auch der Ansicht, das der Einsatz von Chatbot ethisch vertretbar ist.

   





Wie uns das vorletzte Kapitel zeigte gibt es sehr viele Probleme die aus ethischer Sicht mit einhergehen. Was schonmal positiv ist, die Menschen nehmen sich dieser Probleme an. Es fängt klein an, diese Hausarbeit stellt einen ersten kleinen Schritt dar. Es geht weiter mit Prof. Dr. Oliver Bendel, er beschäftigt sich intensiv mit der Maschinenethik. Durch seine Publikationen sollen sich immer mehr Menschen der Thematik bewusst werden. Ganz oben aus unserer Sicht stehen die Ethikkommissionen. Diese Institutionen leisten einen wichtigen Beitrag wenn es um die Ethik in neuen Technologien geht.

Abschließen möchten wir unsere Hausarbeit mit ein Zitat und dessen Bedeutung. 
\glqq Eine menschengerechte Einbindung intelligenter Systeme in hochkomplexe Gesellschaften ist keine individuelle Angelegenheit, sondern eine gesellschaftliche Aufgabe.\grqq\footnote{\cite{BitkomZitat}}
Dieses Zitat finden wir sehr passend. Es darf nicht ein paar Induvidien obliegen, intelligente Systeme zu implentieren. Ferner muss eine Gesellschaft die Handlungsweise des Systems definieren und überwachen. Allerdings liefert uns die Vergangenheit bereits ein negativ Beispiel. \newline
Der Chatbot von Microsoft namens \glqq Tay\grqq\ wurde innerhalb von Stunden zum Rassisten. Doch wie war dies möglich? Der Chatbot war so konzipiert, dass er aus den Benutzereingaben lernte. Das Problem war, dass Tay alles ungefiltert aufnahm was die Benutzer eingaben. So lernte er auch von menschenverachtenden Eingaben. Da sich die Menschen einen Spaß daraus machten Tay mit solchen Texten zu \glqq füttern\grqq, war sein Schicksal besiegelt. Er lernte den Inhalt und gab ihn wieder. So wurde er zum Rassisten. Microsoft musste den Chatbot nach nicht einmal einem Tag abschalten.\footnote{\cite{TaySpiegel}} 

ABSCHLIEßENDER SATZ?!


